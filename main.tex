\documentclass[12pt,-letter paper]{article}

%\usepackage[left=1.5in,right=1in,top=1in,bottom=1in]{geometry}
%\usepackage[left=1.5in,right=1in]{geometry}
%\usepackage{geometry}
%\makeatletter%
%\textheight     243.5mm
%\textwidth      183.0mm
%\textwidth=31pc%
%\textheight=48pc
\usepackage{lipsum}% this package is included to get dummy paragraphs for sample purpose.
\usepackage{ulem}
\usepackage{alltt}
\usepackage{tfrupee}
\usepackage[anticlockwise,figuresright]{rotating}
\usepackage{pstricks}
\usepackage{wrapfig}
\usepackage{pstcol,pst-grad}
 \usepackage{bm}
\usepackage{enumitem}
\usepackage{listings}
    \usepackage{color}                                            %%
    \usepackage{array}                                            %%
    \usepackage{longtable}                                        %%
    \usepackage{calc}                                             %%
    \usepackage{multirow}                                         %%
    \usepackage{hhline}                                           %%
    \usepackage{ifthen}                                           %%
  %optionally (for landscape tables embedded in another document): %%
    \usepackage{lscape}     
    \usepackage{gensymb}     
    \usepackage{tabularx}
\usepackage{ifthen}%
\usepackage{amsmath}%
\usepackage{color}%
\usepackage{float}%
\usepackage{graphicx}%
%\usepackage[right]{showlabels}%
\usepackage{boites}%
\usepackage{boites_exemples}%
\usepackage{graphicx,pstricks}
%\usepackage{enumerate}%
\usepackage{latexsym}
\usepackage[fleqn]{mathtools}
\usepackage{amssymb,amsfonts,amsthm}
\usepackage{mathrsfs,makeidx,listings,verbatim,moreverb}
%%\usepackage{amsthm,mathrsfs,makeidx,listings,verbatim,moreverb}
%\let\eqref\ref%  updated on 20th April 2017

\usepackage{hyperref}%
%\usepackage[dvips]{hyperref}%
\hypersetup{bookmarksopen=false}%
\usepackage{breakurl}%
\usepackage{tkz-euclide} % loads  TikZ and tkz-base

\newcommand{\solution}{\noindent \textbf{Solution: }}
\providecommand{\mbf}{\mathbf}
\providecommand{\rank}{\text{rank}}
\providecommand{\pr}[1]{\ensuremath{\Pr\left(#1\right)}}
\providecommand{\qfunc}[1]{\ensuremath{Q\left(#1\right)}}
\providecommand{\sbrak}[1]{\ensuremath{{}\left[#1\right]}}
\providecommand{\lsbrak}[1]{\ensuremath{{}\left[#1\right.}}
\providecommand{\rsbrak}[1]{\ensuremath{{}\left.#1\right]}}
\providecommand{\brak}[1]{\ensuremath{\left(#1\right)}}
\providecommand{\lbrak}[1]{\ensuremath{\left(#1\right.}}
\providecommand{\rbrak}[1]{\ensuremath{\left.#1\right)}}
\providecommand{\cbrak}[1]{\ensuremath{\left\{#1\right\}}}
\providecommand{\lcbrak}[1]{\ensuremath{\left\{#1\right.}}
\providecommand{\rcbrak}[1]{\ensuremath{\left.#1\right\}}}
\newenvironment{amatrix}[1]{%
  \left(\begin{array}{@{}*{#1}{c}|c@{}}
}{%
  \end{array}\right)
}
\theoremstyle{remark}
\newtheorem{rem}{Remark}
\newtheorem{theorem}{Theorem}[section]
\newtheorem{problem}{Problem}
\newtheorem{proposition}{Proposition}[section]
\newtheorem{lemma}{Lemma}[section]
\newtheorem{corollary}[theorem]{Corollary}
\newtheorem{example}{Example}[section]
\newtheorem{definition}[problem]{Definition}
\newcommand{\sgn}{\mathop{\mathrm{sgn}}}
\providecommand{\abs}[1]{\left\vert#1\right\vert}
\providecommand{\res}[1]{\Res\displaylimits_{#1}} 
\providecommand{\norm}[1]{\left\lVert#1\right\rVert}
%\providecommand{\norm}[1]{\lVert#1\rVert}
\providecommand{\mtx}[1]{\mathbf{#1}}
\providecommand{\mean}[1]{E\left[ #1 \right]}
\providecommand{\fourier}{\overset{\mathcal{F}}{ \rightleftharpoons}}
%\providecommand{\hilbert}{\overset{\mathcal{H}}{ \rightleftharpoons}}
\providecommand{\system}{\overset{\mathcal{H}}{ \longleftrightarrow}}
	%\newcommand{\solution}[2]{\textbf{Solution:}{#1}}
%\newcommand{\solution}{\noindent \textbf{Solution: }}
\newcommand{\cosec}{\,\text{cosec}\,}
\providecommand{\dec}[2]{\ensuremath{\overset{#1}{\underset{#2}{\gtrless}}}}
\newcommand{\myvec}[1]{\ensuremath{\begin{pmatrix}#1\end{pmatrix}}}
\newcommand{\myaugvec}[2]{\ensuremath{\begin{amatrix}{#1}#2\end{amatrix}}}
\newcommand{\mydet}[1]{\ensuremath{\begin{vmatrix}#1\end{vmatrix}}}
\newcommand\figref{Fig.~\ref}
\newcommand\appref{Appendix~\ref}
\newcommand\tabref{Table~\ref}
\newcommand{\romanNumeral}[1]{\uppercase\expandafter{\romannumeral#1}}
%\numberwithin{equation}{section}
%\numberwithin{equation}{subsection}
%\numberwithin{problem}{section}
%\numberwithin{definition}{section}
%\makeatletter
%\@addtoreset{figure}{problem}
%\makeatother

%\let\StandardTheFigure\thefigure
\let\vec\mathbf
\def\inputGnumericTable{}                                 %%
%New macro definitions
\newcounter{matchleft}\newcounter{matchright}

\newenvironment{matchtabular}{%
  \setcounter{matchleft}{0}%
  \setcounter{matchright}{0}%
  \tabularx{\textwidth}{%
    >{\leavevmode\hbox to 1.5em{\stepcounter{matchleft}\arabic{matchleft}.}}X%
    >{\leavevmode\hbox to 1.5em{\stepcounter{matchright}\alph{matchright})}}X%
    }%
}{\endtabularx}

\title{Mathematics}
\date{\today}
\begin{document}
\maketitle{Questions}

\begin{enumerate}
    \item Write the position vector of point which divides the join of points with position vectors  ${3\overrightarrow{a}-2\overrightarrow{b}}$ and ${2\overrightarrow{a}+3\overrightarrow{b}}$ in $2:1$.
    \item Write the number of vectors of unit length perpendicular to both the vectors $\overrightarrow{a} = 2\hat{i} + \hat{j}+2\hat{k}$ and $\overrightarrow{b}=\hat{j}+\hat{k}$.
    \item Find the vector equation of the plane with intercepts $3$,$-$4 and $2$ on $x$, $y$ and $z$-axis respectively.
    
    \item If $x$ $\in $ N and $\mydet{x+3 && -2 \\ -3x && 2x} = 8$, then find the value of $x$.
    \item Use the elementary column operation $C_2 \rightarrow C_2 + 2C_1$ in the following matrix equation : 
    \begin{align*}
        \myvec{2 && 1 \\2&& 0} = \myvec{3 && 1 \\ 2 && 0} \myvec{1 && 0 \\-1 && 1}
    \end{align*}
    \item Write the number of all possible matrices of order $2\times2$  with each entry $1,2 \text{ or } 3$.


    \item Evaluate :
    \begin{align*}
        \int_{0}^\frac{\pi}{2}{\frac{\sin^2x}{\sin x+ \cos x}}{dx}
    \end{align*}
    \item Evaluate:
    \begin{align*}
        \int_{0}^\frac{3}{2}{\mydet{x \cos \pi x}}dx
    \end{align*}
    \item In a game, a man wins \rupee $5$ for getting a number greater than 4 and loses \rupee $1$ otherwise, then a fair die is thrown. The man decided to throw a die thrice but  quit as and when he gets a number greater than $4$. Find the expected value of the amount he wins/loses.
    
    \item A bag contains $4$ balls. Two balls are drawn at random (without replacement) and are found to be white. What is the probability that all balls in the bag are white.
    \item Find :
    \begin{align*}
        \int{\frac{x^2}{x^4+x^2-2}}dx
    \end{align*}
    \item If $x = a \sin 2t (1+\cos 2t)$ and $y = b \cos 2t(1-\cos 2t)$, find $\dfrac{dy}{dx}$ at $t = \dfrac{\pi}{2}$.
    \item Find the coordinates of the point where the line through the points $A$\brak{3, 4, 1} and $B$\brak{5, 1, 6 } crosses the $\mathrm{XZ}$ plane. Also find the angle which this line makes with the $\mathrm{XZ}$ plane.
    \item Find : $\int{\brak{3x + 1} \sqrt{4-3x-2x^2}}dx $
    \item The equation of tangent at \brak{2,3} on the curve $y^2=ax^3+b$ is $y=3x-5$. Find the values of a and b.
    \item A trust invested some money in two types of bonds. The first bond pays $10\%$ interest and second bond pays $12\%$ interest. The trust received \rupee $2,800$ as interest. However, if trust had interchanged money in bonds, they would have got \rupee $100$ less as interest. Using matrix method, find the amount invested by the trust. Interest received on this amount will be given to Helpage India as donation. Which value is reflected in this question ?
    \item Solve the differential equation : 
    \begin{align*}
        y+ x\dfrac{dy}{dx} = x- y\dfrac{dy}{dx}
    \end{align*} 
    \item The two adjacent sides of a parallelogram are $2\hat{i} - 4\hat{j}-5\hat{k}$ and $2\hat{i}+2\hat{j}+3\hat{k}$. Find the two unit vectors parallel to its diagonals. Using the diagonal vectors, find the area of the parallelogram.
    \item Solve the equation for $x$ : $\sin^{-1} x+\sin^{-1}(1-x)=\cos^{-1}x$
    
    \item If $\cos^{-1}\dfrac{x}{a}+\cos^{-1}\dfrac{y}{b}=\alpha$, prove that $\dfrac{x^2}{a^2} - 2\dfrac{xy}{ab}\cos\alpha + \dfrac{y^2}{b^2} = \sin^2\alpha$
    \item Differentiate $x^{\sin x} + \brak{\sin x}^{\cos x}$ with respect to $x$.
    
    \item If $y=\cos(\log x)+3\sin(\log x)$, prove that ${x}^2\dfrac{d^2y}{dx^2} + x\dfrac{dy}{dx}+{y}=0$.
    \item Form the differential equation of the family of circles in the second quadrant and touching the coordinate axes.
    
    \item Prove that the curves $y^2=4x$ and $x^2=4y$ divide the area of square bounded by $x=0$, $x=4$ and $y=0$ into three equal parts.
    \item Show that the binary operation * on $\mathrm{A} = \textbf{R} - \{-1\}$ defines as $a*b = a+b+ab$ for all a, b $\in$ A is commutative and associative of A. Also find the identity element of * in A and prove that every element of A is invertible.
    \item Using properties of determinants, show that $\triangle$ABC is isosceles if:
    \begin{align*}
         \mydet{1 && 1 && 1 \\ 1+\cos A && 1+\cos B && 1+\cos C \\ \cos^2A+\cos A && \cos^2B + \cos B && \cos^2C + \cos C}=0
    \end{align*}
    \item A shopkeeper has $3$ varieties of pens \lq A\rq, \lq B\rq and \lq C\rq. Meenu purchased $1$ pen of each variety for a total of \rupee $21$. Jeevan purchased $4$ pens of \lq A\rq variety, $3$ pens of \lq B\rq variety and $2$ pens of \lq C\rq variety for \rupee $60$. While Shikha purchased $6$ pens of \lq A\rq variety, $2$ pens of \lq B\rq variety and $3$ pens of \lq C\rq variety for \rupee $70$. Using matrix method, find cost of each variety of pen.
    \item There are two types of fertilisers \lq A\rq and \lq B\rq. \lq A\rq consists of $12\%$ nitrogen and $5\%$ phosphoric acid whereas \lq B\rq consists of $4\%$ nitrogen and $5\%$ phosphoric acid. After testing the soil conditions, farmer finds that he needs at least $12$ kg of nitrogen and $12$ kg of phosphoric acid for his crops. If \lq A\rq costs \rupee $10$ per kg and \lq B\rq cost \rupee $8$ per kg, then graphically determine how much of each type of fertiliser should be used so that nutrients requirements are met at a minimum cost.
    
    \item Find the position vector of the foot of perpendicular and the perpendicular distance from the point P with position vector $2\hat{i}+3\hat{j}+4\hat{k}$ to the plane $\overrightarrow{r}.\brak{2\hat{i}+\hat{j}+3\hat{k}} - 26 = 0$. Also find the image of P in the plane.

    \item Prove that the least perimeter of an isosceles triangle in which a circle of radius r can be inscribed is $6\sqrt{3}r$.
    
    \item If the sum of lengths of hypotenuse and a side of a right angled triangle is given, show that area of triangle is maximum, when the angle between them is $\dfrac{\pi}{3}$
    \item Five bad oranges are accidentally mixed with $20$ good ones. If four oranges are drawn one by one successively with replacement, then find the probability distribution of the number of bad oranges drawn. Hence find the mean and variance of the distribution.
\end{enumerate}
\end{document}
\end{document}